\documentclass[12pt]{article}
\usepackage{graphicx}
\usepackage{amsmath}
\usepackage[margin=0.8in]{geometry}
\usepackage{titlesec}
\usepackage{changepage}
\usepackage{graphicx}
\usepackage{setspace}
\usepackage{placeins}
\setstretch{1.25}
\titlespacing*{\section}
{0pt}{1.5ex}{1ex}
\titlespacing*{\subsection}
{2em}{1.2ex}{0.8ex}
\setlength{\parindent}{2em}
\begin{document}

\section{Heat-Exchanger Process Control}

\subsection{Root Locus Sketching}
\begin{adjustwidth}{2em}{0pt}
The open-loop transfer function is given by:
\end{adjustwidth}
\begin{equation}
    G_{OL}(s) = G_c(s) \cdot G_v(s) \cdot G_p(s) = K \cdot \frac{0.02}{4s+1} \cdot \frac{70}{50s+1} 
\end{equation}

\begin{equation}
    G_{OL}(s)\cdot H(s)= \frac{1.4K}{(4s+1)(50s+1)(12s+1)}
\end{equation}
\begin{adjustwidth}{2em}{0pt}
The poles are located at $s_1 = -0.25$, $s_2 = -0.02$, and $s_3 = -0.0833$. Since all the poles are real, there is no need to find $\theta_d$ and $\theta_a$ 
\\ Since there is no zero, there will be 3 branches heading to infinity.
\\ The asymptote centroid $\sigma_a$ and angles $\phi_a$ are:
\end{adjustwidth} 
\begin{equation}
\begin{aligned}
    \\ \sigma_a &= \frac{-0.25 - 0.02 - 0.0833}{3} = -0.1178 \\
    \phi_a &= 60^\circ, 180^\circ, 300^\circ, for\  k = 0,1,2,...
\end{aligned}
\end{equation}

\begin{adjustwidth}{2em}{0pt}
$\sigma_a=180^\circ$ indicates a branch extends to infinity along the real axis, while the asymptote for $\sigma_a=300^\circ$ is simply the symmetry of $\sigma_a=60^\circ$.
\end{adjustwidth}
\begin{equation}
\omega_{d,asym,test}=0.1178tan60^\circ = 0.204
\end{equation}

\begin{adjustwidth}{2em}{0pt}
To find the transfer function, T(s):
\end{adjustwidth}
\begin{equation}
T(s) = \frac{K G(s)}{1 + K G(s) H(s)}
\end{equation}

\begin{equation}
T(s)
= \frac{\dfrac{1.4K}{(4s+1)(50s+1)}}
{1 + \dfrac{1.4K}{(4s+1)(50s+1)(12s+1)}}
\end{equation}

\begin{equation}
T(s) = \frac{1.4K(12s+1)}
{(4s+1)(50s+1)(12s+1) + 1.4K}
\end{equation}

\begin{equation}
T(s)
= \frac{1.4K(12s+1)}
{2400s^3 + 848s^2 + 66s + 1 + 1.4K}
\end{equation}
\newpage
\begin{adjustwidth}{2em}{0pt}
With the characteristic equation, ${2400s^3 + 848s^2 + 66s + 1 + 1.4K}$
\end{adjustwidth}

\[
\begin{array}{c|cc}
s^3 & 2400 & 66 \\
s^2 & 848  & 1 + 1.4K \\
s^1 & 66 - \dfrac{150}{53}(1 + 1.4K) & 0 \\
\end{array}
\]

\begin{equation}
ROZ: 66 - \frac{150}{53}(1 + 1.4K) = 0 => K = 15.9429
\end{equation}

\begin{equation}
1.4K = 22.3201;\ \ 848s^2 + 23.3201 = 0 => s_{cross} = \pm j\,0.1658
\end{equation}

\begin{align}
\frac{d}{ds} (P(s)) &= \frac{d}{ds} \left[ 1.4K \left[ (4s+1)(50s+1)(12s+1) \right]^{-1} \right] = 0 \\
&\frac{d}{ds} \left[ 1.4K (2400s^3 + 848s^2 + 66s + 1)^{-1} \right] = 0 \\
&7200s^2 + 1696s + 66 = 0
\end{align}
\begin{equation}
s_1 = -0.0492 (BAP), \ s_2 = -0.1864(invalid)   
\end{equation}
\begin{adjustwidth}{2em}{0pt} 
With all the information collected, root locus is drawn as shown in Figure 1 and Figure 2 respectively. The root locus will be further processed to design the PD controller in the same figure. The hand-drawn root locus will later be verified through MATLAB simulation.
\end{adjustwidth}
\subsection{Gain Adjustment}
\begin{equation}
    \xi = 0.7,\ \theta_{dom} = \cos^{-1}(0.7) = 45.573^\circ
\end{equation}
\begin{equation}
    let,\ \sigma_{test}=0.06,\ \omega_{d, test}=0.06tan45.573^\circ = 0.06121
\end{equation}
\begin{adjustwidth}{2em}{0pt}
$With\ s_{test} = -0.06 \pm j0.06121$, from the root locus in Figure 1, $s_{d} = -0.046 \pm j0.046$
\end{adjustwidth}


\begin{equation}
    K = \left| \frac{(4s+1)(50s+1)(12s+1)}{1.4} \right|_{s = -0.046 + j0.046} = 1.1222
\end{equation}

Thus, the characteristic equation:
\begin{equation}
    {2400s^3 + 848s^2 + 66s + 2.57108 = 0 => s_{1,2} = -0.0443\pm j0.0457,\  s_3= -0.2647}
\end{equation}
\begin{adjustwidth}{2em}{0pt} 
Since 5 $\omega_{s1,2} = 0.2215 < \omega_{s_3}$, this system could be approximated as a 2nd order system. Thus, 
\end{adjustwidth}
\begin{equation}
    T_s=\frac{4}{\omega_d}=\frac{4}{0.046}=86.9565s
\end{equation}
\begin{equation}
T(s)
= \frac{1.57108(12s+1)}
{2400s^3 + 848s^2 + 66s + 2.5711}
\end{equation}

\newpage
\subsection{1st PD Controller Design}
\begin{adjustwidth}{2em}{0pt} 
The new desired settling time is $20\%$ smaller:
\end{adjustwidth}
\begin{equation}
    T_{s,new} = 86.9565 \times 0.8 = 69.5652 \, \text{s}
\end{equation}
%created by Chan Jia Ming, Zhang Xiaotian, Rasyiqah Aqilah
%All rights reserved
\begin{equation}
    \sigma_{d,new} = \frac{4}{T_{s,new}} = 0.0575
\end{equation}

Using the same damping ratio $\xi = 0.7$:
\begin{equation}
    \omega_{d,new} = 0.0575 \tan(45.573^\circ) = 0.0587
\end{equation}
\begin{adjustwidth}{2em}{0pt}
The new desired pole is $s_{d,new} = -0.0575 \pm j0.0587$.
\end{adjustwidth}

\begin{equation}
    K = \left| \frac{(4s+1)(50s+1)(12s+1)}{1.4} \right|_{s = s_{d,new}} = 1.5412
\end{equation}

Thus, the characteristic equation
\begin{equation}
    2400s^3 + 848s^2 + 66s + 3.15768 = 0 => s_{1,2} =-0.0420\pm j0.0559, s_3 = -0.2694
\end{equation}
\begin{adjustwidth}{2em}{0pt}
Since $5\  \sigma_{1,2} = 0.21 < \sigma_{3}$, this system could be approximated as a 2nd order system. Thus, the settling time formula (19) is applicable.\\ 
\end{adjustwidth}
\begin{adjustwidth}{2em}{0pt}
Applying the angle criterion to find the PD zero location $z_c$:
\end{adjustwidth}
\begin{equation}
    \sum \theta_{poles} - \theta_{zero} = 180^\circ
\end{equation}

\begin{equation}
    \tan^{-1}\left( \frac{0.0587}{0.25 - 0.0575} \right) + \tan^{-1}\left( \frac{0.0587}{1/12 - 0.0575} \right) + \\ \left[ 180^\circ - \tan^{-1}\left( \frac{0.0587}{0.0575 - 0.02} \right) \right] + \theta_{z_1} = 180^\circ
\end{equation}

\begin{equation}
    \theta_{z_1} = 25.777^\circ
\end{equation}
\begin{equation}
    z_1 = -0.0575 - \frac{0.0587}{tan25.777^\circ} = -0.1791
\end{equation}

\begin{adjustwidth}{2em}{0pt}
Hence, $G_D(s) = K_D(s + 0.1791)$
\end{adjustwidth}

\begin{equation}
    G_D(s) \cdot G_{OL} \cdot H(S) = \frac{1.4 K_D (s + 0.1791)}{(4s+1)(50s+1)(12s+1)}
\end{equation}

\newpage
The new gain $K_D$ is:
\begin{equation}
    K_D = \left| \frac{(4s+1)(50s+1)(12s+1)}{1.4(s + 0.1791)} \right|_{s = s_{d,new}} = 11.414
\end{equation}

\begin{equation}
G_{OL}(s)\,G_{P}(s)\,H(s)
= \frac{15.9796(s + 0.1791)}{(4s+1)(50s+1)(12s+1)}
\end{equation}

\begin{equation}
T(s) = \frac{15.9796(s + 0.1791)(12s+1)}{(4s+1)(50s+1)(12s+1) + 15.9796(s + 0.1791)}
\end{equation}

\begin{adjustwidth}{2em}{0pt} 
To be honest it will be very torturing to manually expand T(s), luckily we can quickly obtain the coefficients using the following MATLAB code:

s = tf('s');

num = 15.9796 * (s + 0.1791) * (12*s + 1);

den = (4*s + 1) * (50*s + 1) * (12*s + 1) + 15.9796 * (s + 0.1791);

T = num / den;
disp(T)
\end{adjustwidth}

\begin{equation}
    Hence,\ T(s) = \frac{191.7552s^2 + 50.3230s + 2.8619}{2400s^3 + 848s^2 + 81.9796s + 3.8619}
\end{equation}


\begin{figure}[!htbp]
    \centering
    \includegraphics[width=0.9\textwidth]{Root Locus for the 1st PD Controller.jpg}
    \caption{Root Locus for the 1st PD Controller}
    \label{fig:rl_pd1}
\end{figure}

\FloatBarrier
\subsection{2nd PD Controller Design (Ogata's Method)}


\begin{adjustwidth}{2em}{0pt}
Katsuhiko Ogata's (2010) lead compensator technique is used to explore different potential , where:
\end{adjustwidth}
\begin{equation}
    G_D(s) = K_D \frac{s + \frac{1}{T}}{s + \frac{1}{\alpha T}}, z_c = \frac{1}{T}, \ p_c = \frac{1}{\alpha T}
\end{equation}



\begin{adjustwidth}{2em}{0pt}
Using the new desired pole $s_{d,new} = -0.0575 \pm j0.0587$, which is same as Section 1.3
\end{adjustwidth}

\begin{equation}
\begin{aligned}
    \phi &= \angle [s_{d,new} - (-1/4)] + \angle [s_{d,new} - (-1/12)] + \angle [s_{d,new} - (-1/50)] - 180^\circ \\
    &= 16.9583^\circ + 66.2461^\circ + 122.5721^\circ - 180^\circ \\
    \phi &= 25.7765^\circ \implies \phi/2 = 12.8883^\circ
\end{aligned}
\end{equation}

\begin{adjustwidth}{2em}{0pt}
Using the geometric method with reference points $A, B, C, D, P, O$:
\end{adjustwidth}

\begin{equation}
\begin{aligned}
    \angle PBO &= 180^\circ - 45.57^\circ = 134.43^\circ \\
    B &= \sigma_{d,new} - \frac{\omega_{d,new}}{\tan \theta_{dom}} - \frac{\omega_{d,new}}{\tan \frac{\angle PBO}{2}} = -0.0822
\end{aligned}
\end{equation}

\begin{adjustwidth}{2em}{0pt}
From the geometric construction, the compensator pole and zero are determined based on the graph as shown in Figure 2:
\end{adjustwidth}

\begin{equation}
    C = -0.112, \quad D = -0.057 \implies p_c = 0.112, \ z_c = 0.057, \ \alpha = 1.9649
\end{equation}

\begin{equation}
    Hence,\  G_D(s) = \frac{K_D(s + 0.057)}{(s + 0.112)},\ where \ \alpha = 1.9649
\end{equation}

\begin{adjustwidth}{2em}{0pt}
The gain $K_D$ is calculated at the desired pole $s = s_{d,new}$:
\end{adjustwidth}

\begin{equation}
    K_D = \left| \frac{(s + 0.112)(4s + 1)(50s + 1)(12s + 1)}{1.4(s + 0.057)} \right|_{s = s_{d,new}} = 2.1030
\end{equation}

\begin{adjustwidth}{2em}{0pt}
Given $K = K_D \cdot 1.4 = 2.9442$:
\end{adjustwidth}

\begin{equation}
    G_D(s) \cdot G_OL(s) \cdot H(s) = \frac{2.1030(s + 0.057)}{s + 0.112} \cdot \frac{1.4}{(4s+1)(50s+1)} \cdot \frac{1}{(12s+1)}
\end{equation}

\begin{equation}
    T(s) = \frac{2.9442(s + 0.057)(12s+1)}{(4s+1)(50s+1)(12s+1)(s + 0.112)+2.9442(s + 0.057)}
\end{equation}

\begin{equation}
    T(s) = \frac{35.330s^2 + 4.958s + 0.1678}{2400s^4 + 1116.8s^3 + 160.976s^2 + 11.336s + 0.2798}
\end{equation}

\begin{figure}[!htbp]
    \centering
    \includegraphics[width=0.9\textwidth]{Root Locus for the 2nd PD Controller.jpg}
    \caption{Root Locus for the 2nd PD Controller}
    \label{fig:rl_pd2}
\end{figure}

\subsection{PID Controller Design (Ogata's Method)}

\begin{adjustwidth}{2em}{0pt}
The technique taught in class assumes a PD controller as the ideal integral compensator of the form $G_{PI}(s) = \frac{s + p_c}{s}$. Although it can completely eliminate the $e_{ss}$, it requires require active amplifiers and additional power supplies (N.S. Nise, 2015).  

On the other hand, the lead compensator can be realized through passive circuit, which is more economic and scalable in the industry.

Hence, Ogata's method is used. The work for the derivative part is directly inherited from the previous lead compensator design. To form a PID controller, an integral component $G_I(s)$ is introduced:
\end{adjustwidth}
\begin{equation}
    G_C(s) = K_C \cdot G_I(s) \cdot G_D(s) = K_C\frac{s + \frac{1}{T_2}}{s + \frac{1}{\beta T_2}} \cdot \frac{s + 0.057}{s + 0.112}
\end{equation}

Assume: 
\begin{equation}
     \left| K_C \frac{(s + 0.057)}{s + 0.112} \cdot \frac{1.4}{(4s+1)(50s+1)(12s+1)}\right|_{sd,new} = 1 \implies K_C = 2.1030
\end{equation}
\begin{adjustwidth}{2em}{0pt}
To achieve a steady-state error $e_{ss} = 0.05$, the required position error constant $K_p$ is calculated as:
\end{adjustwidth}
\begin{equation}
    K_p = \frac{1}{e_{ss}} - 1 = \frac{1}{0.05} - 1 = 19
\end{equation}

\begin{equation}
    K_p = \lim_{s \to 0} G_I(s)\cdot G_D(s)\cdot G_{OL}(s)\cdot H(s) = \frac{2.103 \times 0.057 \times 1.4}{0.112}\beta = 19 \implies \beta = 12.6803
\end{equation}

\begin{adjustwidth}{2em}{0pt}
Trial and error were performed to find a value for $T_2$, with the criteria used: $|G_I(s)| \approx 1$ and $-5^\circ < \angle G_I(s) < 0^\circ$ at the desired pole $s_d = -0.0575 \pm j0.0587$. One suitable possibility found is $T_2 = 45$, resulting in:
\end{adjustwidth}

\begin{equation}
    G_I(s) = \frac{s + 0.005}{s + 0.000394}
\end{equation}

\begin{equation}
    G_C(s) \cdot G_{OL}(s) = 2.103 \cdot \frac{s + 0.02222}{s + 0.000176} \cdot \frac{s + 0.057}{s + 0.112} \cdot \frac{1.4}{(4s+1)(50s+1)} \cdot \frac{1}{(12s+1)}
\end{equation}

\begin{equation}
    T(s) = \frac{2.9442(s + 0.02222)(s + 0.057)(12s+1)}{(s + 0.000176)(s + 0.112)(4s+1)(50s+1)(12s+1) + 2.9442(s + 0.005)(s + 0.057)}
\end{equation}

\begin{equation}
T(s) = \frac{35.332s^3 + 5.743s^2 + 0.278s + 0.0037}{2400s^5 + 1117.22s^4 + 161.177s^3 + 11.361s^2 + 0.296s + 0.0009}
\end{equation}
\end{document}
